\section{Úvod}
Myšlenkové mapy představují efektivní nástroj pro vizualizaci a organizaci informací, podporující kreativní myšlení, analýzu problémů a strukturování myšlenek. Jejich využití se rozšířilo do různých oblastí, včetně vzdělávání, projektového řízení, vědeckého výzkumu a osobního rozvoje. Tradičně se myšlenkové mapy vytvářely na papíře, avšak s rozvojem digitálních technologií se stále více uplatňují webové aplikace umožňující dynamickou práci s uzly, propojeními a interaktivními prvky.
\newline

Tato práce se zaměřuje na návrh a implementaci webové aplikace pro tvorbu myšlenkových map, která poskytuje uživatelům intuitivní rozhraní pro vytváření a editaci myšlenkových schémat. Cílem je vytvořit prototyp webové aplikace, která přináší přehledné, čisté uživatelské rozhraní a snadné, zapamatovatelné ovládání.
\newline

Projekt je stavěn na moderních  technologiích pro vývoj webových aplikací, včetně frameworků pro dynamické uživatelské rozhraní a nástrojů pro práci s grafovými strukturami a jejich vizualizaci.
\newline

Jméno projektu Idea-Atlas (z angličtiny: atlas idejí) má vyjadřovat účel projektu; tedy mapovat ideje (myšlenky).

\begin{figure}[h]
    \centering
    \includegraphics[width=0.3\linewidth]{Images/Logo.png}
    \caption{Logo Idea-Atlas}
\end{figure}
\newpage
\subsection{Zadání projektu}
IdeaAtlas bude online webový nástroj pro tvorbu myšlenkových map. Uživatelům umožní jednoduše uspořádat své myšlenky a nápady do grafů závislostí mezi pojmy. Uživatelé budou moci přidávat nové položky, měnit jejich vztahy, mazat je, a tím upravovat myšlenkovou mapu. Další funkcí tohoto nástroje bude generování souvislostí a pojmů pomocí ChatGPT. Každý uživatel bude mít svůj vlastní workspace, což znamená, že jeho mapy budou uloženy na serveru. Tento nástroj bude ideální pro brainstorming a vizualizaci souvislostí libovolné problematiky.
\subsection{Důvod výběru tématu}
Vybral jsem si téma nástroj pro tvorbu myšlenkových map, protože mě zaujal koncept vizuálního mapování nápadů a jejich propojení do dynamické struktury. Zároveň mi téma přišlo obtížné, nikoliv však nemožné. Práce s grafy, interaktivní vizualizací a databázemi přináší spoustu výzev. Dále jsem si chtěl vyzkoušet moderní technologie pro tvorbu webových aplikací a prozkoumat, jak je lze efektivně kombinovat.