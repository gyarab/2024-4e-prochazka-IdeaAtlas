\section{Návrhy na zlepšení}
Ačkoliv je webová stránka ve funkčním stavu, vždy se najde, co vylepšit a jaké nové funkce implementovat. Zde je seznam pár věcí, které jsem nestihl nebo jsem od nich upustil.
\subsection{Personalizace stránky}
Uživatel by si mohl nastavit svůj preferovaný barevný vizuál. Dále by bylo na výběr z několika jazyků; zejména bych rád implementoval český překlad. Uživatelský profil by mohl hrát větší roli, mohl by se upravovat - profilový obrázek, jméno, osobní informace. To by bylo zejména užitečné s implementací popsanou v kapitole: \ref{Realitme}
\subsection{Realtime spolupráce uživatelů}
\label{Realitme}
Tato funkcionalita by umožnila uživatelům sdílet myšlenkové mapy mezi sebou. Mohli by je spolu v reálném čase upravovat. Takováto funkce by udělala můj nástroj ještě užitečnější.
\subsection{ChatGPT api}
Bohužel se mi nepodařilo zařadit ChatGPT API do projektu. ChatGPT API je placená služba a na rozdíl od Supabase nenabízí žádný bezplatný zkušební plán. Nepodařilo se mi najít žádné konkurenční služby, které by to umožnily. Jako další možnost se nabízela cesta vlastního hostování jazykového modelu; to by ovšem výrazně zvětšilo komplexitu projektu a navíc je to velmi výkonnostně velmi náročné.
\subsection{Deploy}
Nasazení (deploy) mého projektu by rozhodně zlepšilo jeho dostupnost a použitelnost. V současnosti je omezený na lokální prostředí, což komplikuje testování a sdílení s uživateli. Pokud bych jej nasadil na produkční server nebo cloudovou platformu, umožnilo by to nepřetržitý přístup odkudkoli, snadnější iteraci a případné škálování podle potřeby. 